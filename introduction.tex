\section{Introduction}
The public cloud networks provide high-speed, reliable, low-latency, and high-performance network services for tenants. This network consists of an extensive global private fiber network with over points of presence (POPs) across the globe. The PoP and data center are interconnected by a private backbone. Leveraging the private backbone, the traffic between the client and the cloud can go through different PoPs. Through the SDN-based centralized control, cloud service providers not only get the management simplicity but also achieve optimization traffic management to reduce the cost based on $95^{th}$ percentile pricing \cite{hong2013achieving,singh2021costCascara}. 

% 差异化诉求
From the perspective of cloud providers, they currently aim to reduce the cost by scheduling the unbalanced traffic among multiple Internet egresses of different regions \cite{yap2017espresso,schlinker2017edgefabric,casado2012fabric}. 
However, modern users are generating more stringent requirements for network services. For example, the finance institution demands ultra-low latency as much as possible globally. The cloud operators have to manually or automatically select the lowest latency path from multiple egress ports between user AS and cloud data centers, which potentially increases the operating cost of cloud networks. 

Therefore, the current cost-oriented scheduling approach can not satisfy the requirement of the growing demands of users. We need a new unified model to define the multiple objectives for both users and cloud providers. Based on the new model, the cloud network operators improve the capability of flow scheduling by guaranteeing the SLA of high-priority services while maintaining low operational costs.

On the other hand, we can further improve the operation effectiveness through joint control of the backbone network and Internet Egress. This is a new trend to realize the consolidated design, such as Microsoft OneWAN \cite{krishnaswamy2023onewan} which realizes unified inter-datacenter TE by considering both backbone and datacetner edges. It focuses on the traffic load balancing, and reliability among the peering edges and the end host in a data center. However, it does not study how to choose peering edges for Internet traffic through the backbone, which is one of the focuses of this paper.

It is challenging to enable performance-aware routing in cloud networks. The continuous growth of traffic as well as the central control of edge and backbone networks lead to high complexity of routing computation. The scale of flows and routing paths increases by O(10) times. % 网络规模和算法规模的数字 

In this paper, we report {\sys}, a performance-aware routing management system. {\sys}’s control plane evaluates multiple policy-compliant candidate paths by measuring their performance systematically for small slices of live traffic. {\sys} then selects the best path choices in the data plane using a carefully designed algorithm, respecting constraints on egress port capacity and performance, and constraints on access to the edge of the Internet across several paths in the backbone. 


Regarding the computational complexity. The optimization consists of two phases: coarse-grained offline optimization and fine-grained online optimization. The offline optimization estimates the bandwidth allocation for each month. The online optimization determines the flow assignment along the path in backbound and egress across the peering edges for each slot (5 mins in the production network).

The contributions of this paper are the following: 
\begin{itemize}[leftmargin=*] 
    \item {\sys} addresses the key problem of large-scale traffic scheduling in cloud network which serves both Internet and inter-datacenter traffic using software-defined control and proactively route traffic based on the variant users' demands (delay, loss rate, throughput, and cost). 
    \item  With the order of magnitude increase in network and routing table sizes. We developed a new egress scheduling paradigm to manage the increased network scale using existing network equipment. We introduce the techniques to find paths in the network through an optimization solver
    \item  We verify the capabilities of the proposed approaches in terms of the increased matrix size and compute traffic allocations within a few seconds of scheduling in the control plane and data plane in practice.
\end{itemize}

About the complicated control and management of egress and backbone networks. To provide network service based on users' requirements and budget. We first empower them to optimize their cloud network for performance or price. The network performances impact sales, user experience, and user engagement. While on-site optimization is important for improved speed, it is not the only aspect we should look at. The hardware and network infrastructure supporting our website and connecting it to our visitors also matter. In today’s race to speed up website loading time, every millisecond matters. 

